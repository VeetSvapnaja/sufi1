\chapter{Sukuntum Bunga Teratai di Danau Kehampaan}

\bahasa
26 Agustus 1977 pagi di Aula Buddha

\english
26 August 1977 am in Buddha Hall

\bahasa
Pertanyaan pertama:

\english
The first question:

\bahasa
Pertanyaan 1

\english
Question 1

\bahasa
MENGAPA KITA MELUPAKAN KEILAHIAN KITA? APAKAH ARTINYA INI?

\english
Why do we forget our divinity? what is the meaning of this?

\bahasa
Engkau belum melupakannya, engkau belum pernah mengetahuinya - jadi bagaimana engkau dapat melupakannya? Kelupaan hanya mungkin setelah engkau mengetahuinya - dan sekali diketahui, itu tidak pernah terlupakan.

\english
You have not forgotten it, you have never known it - so how can you forget it? Forgetfulness is possible only once you have known it - and once known, it is never forgotten.

\bahasa
Engkau adalah ilahi tapi engkau belum mengetahuinya. Sebenarnya, itu karena engkau ilahi sehingga sangat sulit untuk mengetahuinya. Itu adalah inti dari keberadaanmu. Jika itu adalah sesuatu di luar dirimu, engkau pasti akan menemuinya sekarang. Jika itu adalah sesuatu yang objektif, engkau dapat melihatnya. Tapi itu bukan di luar dan itu bukan obyek - itu adalah subjektivitasmu. Itu bukan sesuatu yang harus dilihat, itu tersembunyi di si pelihat. Itu adalah sebuah penyaksian. Kecuali engkau berada di belakang dirimu sendiri, engkau tidak akan dapat mengetahuinya.

\english
You are divine but you have not known it yet. In fact, it is because you are divine that it is so difficult to know it. It is at the very heart of your being. If it were something outside of you, you would have encountered it by now. If it were something objective, you could have seen it. But it is not outside and it is not an object - it is your subjectivity. It is not something to be seen, it is hidden in the seer. It is a witnessing. Unless you go behind yourself you will not be able to know it.

\bahasa
Ada tiga hal di dunia ini. Pertama adalah dunia objek - hal-hal yang mengelilingimu. Lebih dekat dengan itu, ketika engkau mendatangi dirimu sendiri, adalah dunia pemikiran, impian dan keinginan. Itu juga mengelilingimu. Itulah yang biasanya engkau sebut dunia batin. Itu tidak di dalam, itu masih luar. Ada dua macam sisi luar. Yang engkau lihat dengan mata terbuka dan yang engkau lihat dengan mata tertutup. Tapi keduanya ada di luar karena apapun yang dapat dilihat harus di luar. Untuk dilihat, itu harus di luar, itu harus berbeda dari dirimu. Objeknya harus berbeda dari subjek.

\english
There are three things in the world. One is the world of the object - the things that surround you. Closer to it, when you come towards yourself, is the world of thoughts, dreams and desires. That too surrounds you. That's what you ordinarily call the inner world. It is not inner, it is still outer. There are two kinds of outsides. One that you see with open eyes and one that you see with closed eyes. But both are outside because whatsoever can be seen must be outside. To be seen, it has to be outside, it has to be different from you. The object has to be different from the subject.
